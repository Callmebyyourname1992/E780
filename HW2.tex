\documentclass[12pt]{article}

\usepackage[utf8]{inputenc}


\usepackage{amsmath, amssymb, amsfonts, amsthm, mathrsfs, bm}
%\usepackage{array, multirow}
\usepackage{color, float, graphicx, caption, subcaption}
\usepackage{comment}

\usepackage{mathtools}%for paired delimiters

\usepackage{hyperref}

\renewcommand{\arraystretch}{1.0}

\captionsetup{font=small}
\usepackage{setspace}
\setstretch{1.0}
%\setlength\topmargin{0in}
%\setlength\textheight{8in}
%\setlength\textwidth{5.5in}
%\setlength\evensidemargin{.5in}
%\setlength\oddsidemargin{.5in}

\setlength\topmargin{-.45in}
\setlength\textheight{8.5in}
\setlength\textwidth{6in}
\setlength\evensidemargin{.25in}
\setlength\oddsidemargin{.25in}


\graphicspath{{InkscapePics/}}

\newcommand{\executeiffilenewer}[3]{%
 \ifnum\pdfstrcmp{\pdffilemoddate{#1}}%
 {\pdffilemoddate{#2}}>0%
 {\immediate\write18{#3}}\fi%
}
\newcommand{\includesvg}[1]{%
 \executeiffilenewer{#1.svg}{#1.pdf}%
 {inkscape -z -D --file=#1.svg %
 --export-pdf=#1.pdf --export-latex}%
 \input{#1.pdf_tex}%
}


\DeclarePairedDelimiter\ceil{\lceil}{\rceil}


\renewcommand{\a}{\alpha}
\renewcommand{\b}{\beta}
\renewcommand{\d}{\delta}
\newcommand{\e}{\varepsilon}
\newcommand{\g}{\gamma}
\renewcommand{\k}{\kappa}
\renewcommand{\l}{\lambda}
\newcommand{\m}{\mu}
\renewcommand{\o}{\omega}
\newcommand{\p}{\rho}
\newcommand{\s}{\sigma}
\renewcommand{\t}{\tau}
\renewcommand{\th}{\theta}
\newcommand{\z}{\zeta}


\newcommand{\ba}{\bm{a}}
\newcommand{\bg}{\bm{\gamma}}
\newcommand{\bp}{\bm{p}}
\newcommand{\bq}{\bm{q}}
\newcommand{\bmm}{\bm{m}}
\newcommand{\bmu}{\bm{\mu}}
\newcommand{\bnu}{\bm{\nu}}
\newcommand{\bw}{\bm{w}}
\newcommand{\bx}{\bm{x}}
\newcommand{\bpi}{\bm{\pi}}
\newcommand{\bs}{\bm{\sigma}}
\newcommand{\bS}{\bm{\Sigma}}
\newcommand{\bth}{\bm{\theta}}
\newcommand{\bu}{\bm{u}}
\newcommand{\bv}{\bm{v}}

\newcommand{\bA}{\bm{A}}
\newcommand{\bH}{\bm{H}}
\newcommand{\bM}{\bm{M}}
\newcommand{\bW}{\bm{W}}
\newcommand{\bX}{\bm{X}}

\newcommand{\rp}{{_{r}p}}
\newcommand{\rD}{{_{r}\D}}
\newcommand{\sD}{{_{s}\D}}



\newcommand{\A}{\mathcal{A}}
\newcommand{\C}{\mathcal{C}}
\newcommand{\D}{\Delta}
\newcommand{\mD}{\mathcal{D}}
\newcommand{\E}{\mathcal{E}}
\newcommand{\F}{\mathcal{F}}
\newcommand{\G}{\mathscr{G}}
\renewcommand{\H}{\mathcal{H}}
\newcommand{\I}{\mathcal{I}}
\newcommand{\M}{\mathcal{M}}
\renewcommand{\O}{\Omega}
\newcommand{\Pt}{\mathscr{P}}
\renewcommand{\P}{\mathcal{P}}
\newcommand{\Q}{\mathcal{Q}}
\newcommand{\R}{\mathbb{R}}
\renewcommand{\S}{\mathcal{S}}
\newcommand{\T}{\mathcal{T}}
\newcommand{\Th}{\Theta}
\newcommand{\Y}{\mathcal{Y}}
\newcommand{\V}{\mathscr{V}}
\newcommand{\Z}{\mathbb{Z}}


\newcommand{\ol}{\overline}
\newcommand{\oll}[1]{\overline{\overline{ #1}}}
\newcommand{\ul}{\underline}
\newcommand{\Ex}{\mathbf{E}}
\renewcommand{\Pr}{\mathbf{P}}
\newcommand{\td}{\tilde}
\newcommand{\tr}{\triangleleft}

\newtheorem{lemma}{Lemma}
\newtheorem*{example}{Example}
\newtheorem{theorem}{Theorem}
\newtheorem{proposition}{Proposition}
\newtheorem*{definition}{Definition}
\newtheorem*{maxminrefinement}{Max-Min Threshold Refinement}
\newtheorem*{multimaxminrefinement}{Multi-Worker Max-Min Threshold Refinement}
\newtheorem*{refinement}{An Intuitive Credible Threats Refinement}
\newtheorem*{minrefinement}{Minimal Refinement}
\newtheorem{corollary}{Corollary}
\newtheorem{observation}{Observation}
\newtheorem{remark}{Remark}
\newtheorem{assumption}{Assumption}
\newtheorem*{efficiency}{Efficiency Wage Contract}

\DeclareMathOperator*{\argmax}{arg\,max}
\DeclareMathOperator*{\argmin}{arg\,min}
\DeclareMathOperator{\sgn}{sgn}



\begin{document}


\begin{center}
{\Large Econ 780 \hspace{0.5cm} HW Week 2}\\
\textbf{Minh Cao, Grant Smith, Ella Barnes, Alexander Erwin and Mark Coomes}\\ %You should put your name here
Due:  %You should write the date here.
\end{center}

\vspace{0.2 cm}


\subsection*{Exercises for chapter 3}

\begin{enumerate}




\item Exercise 1






\item Exercise 2.\\
 Consider augmented the matrix.\\
 
  $\ol{D}=
    \left[\begin{array}{llll}
    1 & 1 & 2 & 3\\
    \end{array}
    \right].$ 
    \\
    
    There is no arbitrage. since there only 1 matrix, so the only choice to make the value of the profolio is zero in day zero is buy nothing, but it will turnout that the value of the porlio at day 1 will be zero. So there will be no arbitrage opportunity.\\
    
    let $Q(w_{1}), Q(w_{2}), Q(w_{3})$be Martingale measure for the market.\\
    We have the formular.\\
    
    $$1 = \beta(Q(w_{1}+2Q(w_{}{2})+3Q(w_{3}))$$ and\\ 
 



$$Q(w_{1}) +  Q(w_{2})+ Q(w_{3}) = 1$$. \\ 
For some $\beta$\\
For simplicity, I will use $Q_{1}, Q_{2}, Q_{3}$ instead of $Q(w_{1}), Q(w_{2}), Q(w_{3})$

$$1 = \beta(Q(w_{1}+2Q(w_{}{2})+3Q(w_{3})) \implies \frac{1}{\beta} = Q_{1} + 2 Q_{2} +3 Q_{3}$$\\
Let write $Q_{1} = 1-Q_{2}-Q_{3}$ and substitiue to the equation above, we obtain:\\
$$\frac{1}{\beta} = 1-Q_{2}-Q_{3} + 2 Q_{2} +3 Q_{3} = 1+Q_{2}+2Q_{3} \implies Q_{2} =\frac{1}{\beta} - 1 - 2Q_{3} $$
again, substitute $Q_{2} = 1-Q_{1}-Q_{3}$ Obtain\\
$$\frac{1}{\beta} = Q_{1} + 2(1-Q_{1}-Q_{3}) +3 Q_{3}  = 2-Q_{1}+Q_{3} \implies Q_{1} = 2-\frac{1}{\beta}+Q_{3}$$.\\ Now we will find the bounds for $Q_{2}$ to be legitimate probability measure.
\begin{itemize}
\item $0<Q_{1}<1 \implies 0<2-\frac{1}{\beta}+Q_{3}<1 \implies \frac{1}{\beta} -2 < Q_{3} < \frac{1}{\beta} - 1$
\item $0<Q_{2}<1 \implies 0<\frac{1}{\beta} - 1 - 2Q_{3} <1 \implies \frac{\frac{1}{\beta} - 2}{2} < Q_{3}<\frac{\frac{1}{\beta}-1}{2}$\\
Combine these two inequalities.\\ 
$$\frac{1-2\beta}{2\beta} < Q_{3}<\frac{1-\beta}{2\beta}$$
Hence the family of martingale measure can be writte as:\\
$$\{2-\frac{1}{\beta}+Q_{3},\frac{1}{\beta} - 1 - 2Q_{3} ,Q_{3} \mid \frac{1-2\beta}{2\beta} < Q_{3}<\frac{1-\beta}{2\beta} \}$$


\item Consider $\ol{D}=
    \left[\begin{array}{llll}
    1 & 1.2 & 1.2 & 1.2 \\
    2 & 2.4 & 2.5 & 2.4
    \end{array}
    \right]$.
   
  Let transfer the market to nomarlized market Z:
  
  $D^{Z} =
 \left[\begin{array}{llll}
    1 & 1& 1 & 1 \\
    2 & 2& \frac{2.5}{1.2} & 2
    \end{array}
    \right]$..\\
    
Here asset 1 is risk-free asset with the interest rate i $0$.\\
let $\alpha_{1}, \alpha_{2}, \alpha_{3} $strictly positive and $\alpha_{1} +\alpha_{2}+ \alpha_{3}=1$
the equation:\\
$2 = 2\alpha_{1} +\ frac{25}{12}\alpha_{2} + 2(1- \alpha_{1} -\alpha_{2}) \implies \alpha_{2} = 0$  a contradiction. hence the market is not arbitrage free.\\
% We dont need this part for this problem, just point out the arbitrage porfolio.

In fact, in the original market. Let consider the portfolio $h= (-2,1)$, hence ${V_{0}}^{h}= -2+2 = 0$ and $P({V_{1}}^{h} \geq 0) = 1, P({V_{1}}^{h} >0) >0$ Q.E.D
 


\end{itemize}•



































\end{enumerate}




























\end{document}


